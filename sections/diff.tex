\section{Differential Calculus}
\subsection{Formal Definition of a derivative}
\begin{defn}[Limit Definition of a derivative]
	The derivative of a continuous function $f(x)$ on an interval $I$ is defined as the limit of the difference quotient
	$$f'(x) = \lim_{\Delta x \to 0}{\frac{f(x + \Delta x) - f(x)}{\Delta x}}$$
	Where $\Delta x$ is a small change in $x$
\end{defn}
\subsection{Derivative Rules and Properties}
The differntial operator is a linear operator, i.e.\,
$${(f(x) \pm g(x))}' = f'(x) \pm g'(x)$$
$${(c\cdot f(x)}' = c\cdot f'(x)$$
\paragraph{Product Rule}
$$(f(x)\cdot g(x))' = f'(x)\cdot g(x) + f(x)\cdot g'(x)$$
Quotient Rule:
$$\left(\frac{f(x)}{g(x)}\right)' = \frac{f'(x)\cdot g(x) - g'(x)\cdot f(x)}{(g(x))^2}$$
\paragraph{Chain Rule}
$$ (f(g(x)))' = f'(g(x))\cdot g'(x)$$
\begin{thrm}[Leibnitz' Theorem]
	For a function $f(x) = u(x)v(x)$, the ${n^{th}}$ derivative, $f^{(n)}(x)$ is given by
	$$f^{(n)}= \sum_{r=0}^{n}{^nC_r u^{(r)}v^{(n-r)}}$$
\end{thrm}
\subsection{Common Derivatives}
\begin{align*}
&\diff{x}a^x = (\ln a)a^x & &\diff{x}\log_a x = \frac{1}{x\ln a} \\
&\diff{x}\sin x = \cos x & &\diff{x}\cos x = -\sin x & &\diff{x}\tan x = \sec^2 x\\
&\diff{x}\csc x = -\csc x \cot x & &\diff{x}\sec x = \sec x \tan x & &\diff{x}\cot x = -\csc^2 x \\
&\diff{x}\sin^{-1}x = \frac{1}{\sqrt{1 - x^2}} & &\diff{x}\cos^{-1}x = - \frac{1}{\sqrt{1 - x^2}} & &\diff{x}\tan^{-1}x = \frac{1}{1 + x^2} \\
&\diff{x}\sec^{-1}x = \frac{1}{ x \sqrt{x^2 - 1}} &  &\diff{x}\csc^{-1}x = -\frac{1}{ x \sqrt{x^2 - 1}} & &\diff{x}\cot^{-1}x = -\frac{1}{1 + x^2} \\
\end{align*}
\subsection{Mean Value Theorem}
\begin{thrm}[Mean Value Theorem]
	If a function $f(x)$ is continuous and differentiable in the range $(a,c)$, then there exists atleast one value b, $a < b < c$, such that
	$$f'(b) = \frac{f(c) - f(a)}{c - a}$$
\end{thrm}
\subsection{Applications of Derivatives}
\subsubsection{Tangent to a Curve}
Tangent to a curve $f(x)$ at a point $a$:
$$\diff{x}f(x)\Big\rvert_{x=a}$$
For a straight line passing through points $\mathrm{(x_1,y_1)}$ and  $\mathrm{(x_2, y_2)}$, the slope, m, is constant and is calculated by:
$$m = \tan\theta = \frac{y_2 - y_1}{x_2 - x_1}$$
Where $\theta$ is the angle of the line with the x-axis

\subsubsection{Analysis of a Curve}
\paragraph{Critical Points}
$x = a$ is a critical point of $ f(x) $ if $ f'(a)=0 $ or $ f'(a) $ doesn't exist.
\paragraph{Slope}
\begin{enumerate}
	\item $ f(x) $ is increasing on an interval $I$ if $f'(x) > 0$, i.e.\ it has a positive slope on that interval.
	\item $f(x)$ is decreasing on an interval $I$ if $f'(x) < 0$, i.e.\ it has a negative slope on that interval.
	\item $f(x)$ is constant on an interval $I$ if $ f'(x) = 0 $.
\end{enumerate}
\paragraph{Concavity}
\begin{enumerate}
	\item $f(x)$ is concave up on an interval $I$ if $ f''(x) > 0 $.
	\item $f(x)$ is concave down on an interval $I$ if $f''(x) < 0 $.
\end{enumerate}
\paragraph{Inflection Points}
$x = a$ is an inflection point of $f(x)$ if the concavity changes at $f(a)$.
\paragraph{Extrema}
$f(a)$ is a stationary point on an interval $I$ if $f'(a) = 0$.
\begin{enumerate}
	\item If $f''(a) > 0$, then $f(a)$ is a local minimum.
	\item If $f''(a)< 0$, then $f(a)$ is a local maximum.
	\item If $f''(a)= 0$, then the second derivative test fails.
\end{enumerate}

\subsubsection{Taylor Polynomials}
\begin{defn}[Taylor Polynomial]
\label{taylorpoly}
Let $f(x)$ be a real-valued function that is infinitely differentiable at $x = x_0$. The Taylor series expansion for the function $f(x)$ centered around the point $x = x_0$ is given by
$$\sum_{n=0}^{\infty}f^{(n)}(x_0)\frac{(x - x_0)^{n}}{n!}$$
Where $f^{(n)}(x_0)$ is the $n^\text{th}$ derivative of $f(x)$ at $x = x_0$.
\end{defn}
\subsection{Partial Derivatives}